%!TEX root = ./main.tex
%
% This file is part of the i10 thesis template developed and used by the
% Media Computing Group at RWTH Aachen University.
% The current version of this template can be obtained at
% <http://www.media.informatik.rwth-aachen.de/karrer.html>.
%\documentclass[oneside]{book}
\documentclass[11pt, a4paper, titlepage]{book}
\usepackage{graphicx} %package to manage images
\graphicspath{ {images/} }
% ! ACHTUNG !
% Nach dem ersten LaTeX durchlauf auf der Kommandozeile
% makeindex -s main.ist main.idx
% ausf�hren, sonst wird der Index nicht so sch�n formatiert.
% Leider macht TeXShop den makeindex Aufruf nur ohne Parameter.

% !ATTENTION!
% After running LaTeX for the first time using this template call
% makeindex -s main.ist main.idx
% to get the correct formatting for the index.

%Pakete und eigene Befehle - am besten durchlesen!
%Includes all neccessary packets and contains some useful commands - read this file!
%\include{i10preamble}
\usepackage{url}
\usepackage{listings}
\lstset{
	numbers=left,
	numberstyle=\tiny,
	numbersep=5pt,
	xleftmargin=11pt,
	xrightmargin=4pt,
	frame=single,
	aboveskip=0pt,
	belowskip=-6pt,
	sensitive=true,
	float=!t,
	breaklines=false,
	captionpos=b,
	tabsize=2,
	showstringspaces=false,
	basicstyle=\small\ttfamily,
	morecomment=[l]{//},
	morecomment=[s][\itshape]{/**}{*/}
}

\setlength{\voffset}{-1in}
\setlength{\hoffset}{-1in}

\setlength{\topmargin}{2.5cm}		   
\setlength{\headheight}{0cm}		   
\setlength{\headsep}{1cm}		   
\setlength{\oddsidemargin}{3.3cm}  % innen ein wenig mehr Rand f�r die Klebebindung
\setlength{\evensidemargin}{2.7cm} % daf�r au�en ein wenig weniger
\setlength{\textwidth}{15cm}		   
\setlength{\textheight}{23,5cm}		   
\setlength{\parindent}{0cm}
%--------------------------------------------------------------
%Dokumentspezifisches
%Stuff regarding your specific document
%--------------------------------------------------------------

%Trennungshilfen
%Hyphenation patterns
\hyphenation{
dieseswortwirdnichtgetrennt
diesesauchnicht
thiswordwillstayinoneline
thistoo
}

\usepackage{graphicx}

%--------------------------------------------------------------
\begin{document}

% gr��ere Wortabst�nde zulassen, um Trennungen zu vermeiden
% allow more flexible whitespaces to avoid hyphenation and overfull hboxes


% "see" Eintr�ge f�r den Index
% 'see'-entries for the index
\include{indexAddOn}


%--------------------------------------------------------------
\frontmatter

%Die Titelseite aus dem "images"-Verzeichnis wird verwendet
%use the titlepage from the 'images' directory
%Bachelor's: titlepage_Bachelor
%Master's:   titlepage_Master
%Neutral:    titlepage
% Dieses Teildokument beschreibt die Titelseite.
%

% Seitenz�hler auf 1, R�mische Ziffern.
\setcounter{page}{1}
\pagenumbering{roman}

\thispagestyle{headings}

% Eigentliche Titelseite.
\begin{titlepage}
	\begin{center}
		{\LARGE
			Rheinisch Westf�lische Technische Hochschule Aachen \\
			Informatik 5 (Information Systems) \\
			\vspace{4em}
			
			\vspace{1em}
			
			\begin{figure}[h!]\begin{center}\resizebox{20em}{!}{\includegraphics{rwthlogo}}\end{center}\end{figure} 
			
			\vspace{1em}
			\begin{figure}[h!]\begin{center}\resizebox{10em}{!}{\includegraphics{i5-logo}}\end{center}\end{figure} 
			
			\vspace{1em}		
			
			{\LARGE \bf Master Thesis Proposal}\\ [1em]
			\bf Semantic Data Profiling in Data Lake for Cancer Genome  \par} 
		
		\vspace{4em} 
		
		{\large \lineskip 1em
			
			\begin{tabular}[t]{rl}
				\multicolumn{2}{c}{\LARGE \bf Jasim Waheed Ansari}\\ 
				&\\
				{\Large \bf Supervisor:}&{\Large  Prof.\ Dr.\ Stefan \ Decker}\\[.5em]
				{\Large \bf Advisors:}&{\Large Dr. Oya Deniz Beyan, \Large Dr. Michael Cochez, \Large Naila Karim}\\
			\end{tabular}
			\par}
		
		\vspace{4em}
		{\Large	Aachen, den \today \par}
	\end{center}
\end{titlepage}

% \thispagestyle{empty}
% \emptydoublepage

% \include{declarationofownwork}
% \emptydoublepage

\setcounter{page}{5}

\tableofcontents
\cleardoublepage
%
% \listoffigures
% \emptydoublepage
%
% \listoftables
% \emptydoublepage

%Die �bersicht sollte in englischer und in deutscher Sprache verfasst werden
%the abstract should include an english and a german version
% \include{abstract}
% \emptydoublepage

%Danksagungen
%acknowledgements
% \include{acknowledgements}
% \emptydoublepage

%In der Arbeit verwendete Konventionen (Formelsatz, Definitionen, etc.)
%conventions applied in the thesis (AE/BE, definitions, etc.)
%\include{conventions}
%\emptydoublepage

%--------------------------------------------------------------
\mainmatter

%!TEX root = ./main.tex
%
% This file is part of the i10 thesis template developed and used by the
% Media Computing Group at RWTH Aachen University.
% The current version of this template can be obtained at
% <http://www.media.informatik.rwth-aachen.de/karrer.html>.

\chapter{Introduction}
Cancer is one of the leading causes of death around the globe. It is considered to be the most complex disease that human species has to deal with. It is characterized by undesirable growth in cells which are caused by changes in genomes and exposure to environment\cite{5}. Each cancer type has unique architecture of genetic variation - somatic mutations, copy number alterations, gene expression profiles and different epigenetic aberration. Therefore, a strong need for better diagnosis, personalized medicines for patients have arose, which correlates with better understanding of changes at genetic level in tumors. All thanks to technological advancements in high-throughput techniques for genome-wide interrogations such as Microarray and Next-Generation Sequecing(NGS) platforms have provided wealth of multi-disciplinary, multi-institutional pilot projects in cancer studies\cite{stratton2009cancer} \cite{lengauer1998genetic}.\\~\\
Various unique projects catering to cancer genome study such as 1000 Genome Project\cite{10002010map}\cite{10002012integrated}, Encyclopedia of DNA Elements Project (ENCODE)\cite{ecker2012genomics}, Immunological Genome Project (ImmGen)\cite{shay2013immunological}, The Cancer Genome Atlas (TCGA)\cite{lee2016exploring} which aims at investigating biological systems at several levels through the creation of humongous datasets with different data types. These projects motivates the bioinformaticians such as development of novel data integration methodologies to explore hidden insights.\\~\\ 
With respect to current state-of-the-art for data integration platforms, data lake promises to addresses the onslaught of Big Data generated across the projects\cite{fang2015managing}. However, one of the biggest challenges in data lake is to prevent turning into a data swamp or silos. Which means, the lake turns into a set of disconnected data pools because of lack of semblance of information governance. Presently, 70\%  of the time spent in data analytics projects is into informational discovery of the data - identification, location, integration and re-engineering\cite{terrizzano2015data}. Moreover, especially for cancer genomic data, where the users must have a first hand knowledge in biomedical area since its metadata alone cannot provide the full meaning out of respective datasets. There are dearth of novel methodologies that can point out, for example - genes with name "TP53" and "BCC7" or disease with name "Breast Cancer" and "Breast Tumor" means the same. Additionally, in context of conceptual hierarchy, the patient's clinical data set has a broader term as a Person or narrower terms such as Outgoing, Incoming or Doner.  Thus, the use of knowledge organization system from both biomedical domain as well as generic is needed to understand the data more unambiguously.  This would assist the users to find, explore, understand, and trust the data at ease. Such domain knowledge and thesauri are captured as specification of conceptualization(or commonly known as ontologies) in BioPortal\cite{noy2009bioportal} and Simple Knowledge Organization Systems(SKOS)\cite{miles2005skos} etc. \\~\\
Keeping in consideration of our focus of work in relation to semantic discovery and management of metadata around the informational content, we focus on profiling aspect. But, existing profiling capabilities revolves around catching statistical summaries of anomalies and does not offer semantic aspect. This motivates us to an idea of enabling semantic profiling in data lakes. \\~\\
Semantic Profiling is a methodology that exploits semantic-based tools and ontologies in order to derive lucid understanding of the information being stored in current systems. This approach is much more rigorous than former traditional profiling techniques for the following reasons: \\
i) Unlike traditional profiling which captures as much anomalies as you can until you stop, whereas, in semantic profiling once you select the domain of study, you stick by the rules and guidelines. \\
ii) If a user perform semantic profiling on one system and would like to combine the results of first system  with those from other system, it can easily be facilitated through its feature of reusability. \\
iii) Semantic profiling allows setting up of testing hypothesis to monitor the data lake in production scenario which can detect semantic drift.\\~\\

Therefore, through this thesis work, we extend current profiling efforts in the data lake solutions in a much more rigorous way i.e. through semantical enrichment using ontologies in Figure \ref{fig:Figure1}. This would open the door in answering questions about data sources like: 
\begin{itemize}
	\item Does the biological data adhere to particular standards such as gene and disease ontologies?
	\item  Is there a way to capture controlled vocabularies, taxonomies and thesauri for clinical information of patients?
\end{itemize}
\begin{figure}[htbp]
	% center the image.
	\centering
	
	% include a png file. Adapt size to 0.5 * textwidth and retain aspect ratio (!)
	%	\resizebox{\textwidth}{!}{Figure1.png}}
	% NOTE: if possible do not include bitmap graphics in your paper, if available use
	% a vector graphics format
	\includegraphics[scale=0.3]{./images/datalake.JPG}
	\caption{Need to extend the current semnantic tendencies of data lake}
	\label{fig:Figure1}
\end{figure} \newpage
\section{Thesis Goals \& Outcome}
\label{goals}
Specific aims of my thesis are as follows:
\begin{itemize}
	\item Extend and develop current profiling efforts in our state-of-the-art data lake with:
	\subitem{a)} a systematic process of annotating the ingested data's schema\cite{bernstein2011generic}
	\subitem{b)} a systematic approach of extracting, managing and exploiting metadata of the datasets' information with  ontologies(for example, BioPortal, SKOS) through ontology alignment techniques\cite{romero2009ontology}
 \item Analysis of the results
 \subitem{a)} Develop a dashboard containing provenance to visually debug the quality of resulting data
\end{itemize}

\clearpage
%!TEX root = ./main.tex
%
% This file is part of the i10 thesis template developed and used by the
% Media Computing Group at RWTH Aachen University.
% The current version of this template can be obtained at
% <http://www.media.informatik.rwth-aachen.de/karrer.html>.

\chapter{Background \& Related work}
\label{relatedwork}
\section{What is Cancer Genome?}
\section{Current state-of-art}
\subsection{The Cancer Genome Atlas}
\label{relatedwork.tcga}
\subsubsection{cBioPortal}
\label{relatedwork.cbioportal}
\subsubsection{Firebrowse}
\label{relatedework.firebrowse}
\section{Big Data Fundamentals}
\label{relatedework.fundamentals}
\subsection{Technologies for Big Data}
\label{relatedework.technologies}
\subsection{Shortcomings in traditional Data Warehousing}
\label{relatedework.tdw}
\subsection{Need for Data Lakes in Big Data}
\label{relatedework.datalake}
\subsection{Big Data Visualization}
\label{relatedework.visualization}


\clearpage
%!TEX root = ./main.tex
%
% This file is part of the i10 thesis template developed and used by the
% Media Computing Group at RWTH Aachen University.
% The current version of this template can be obtained at
% <http://www.media.informatik.rwth-aachen.de/karrer.html>.

\chapter{Solution}
\label{ownwork} 
\section{Extend and develop current profiling efforts in Kylo, a data lake solution}
\label{ownwork.bigdatalake}

\subsection{Systematic process of annotating the ingested data's schema\cite{bernstein2011generic}}
\label{ownwork.nifi}
\subsection{Systematic approach of extracting, managing and exploiting metadata of the datasets' information with  ontologies
}
 \section{Analysis of the results}
 In order to make claims about our developed methodology is fool-proof, we need to analyze the results through work-flows In this work, we utilize Provenance as an infrastructure for debugging and consistency check on our results.
\subsection{Develop a dashboard containing provenance to visually to visually debug the quality of resulting data}
We perform in-depth analysis of the semantically annotated result through visualization in a dashboard, taking into two aspects in consideration: \\~\\
a) Tracking our current results against the newly scheduled run on the data source \\
b) Tracking if the process is identical from two different data stewards

\label{ownwork.standardization}


\clearpage
%!TEX root = ./main.tex
%
% This file is part of the i10 thesis template developed and used by the
% Media Computing Group at RWTH Aachen University.
% The current version of this template can be obtained at
% <http://www.media.informatik.rwth-aachen.de/karrer.html>.

\chapter{Evaluation}
This section is to guarantee that the results we achieved through our work is not random result but meeting our thesis objectives. \\~\\
The evaluation will be carried out using the Broad GDAC Firehose dataset into two separate categories\cite{https://doi.org/10.7908/c11g0km9}: \\
a) molecular datasets \\
b) clinical datasets \\~\\
It is to be noted that the stability of the results is very important. Hence, we will test the feasibility of our work  against previous methodologies on the same dataset.

\clearpage
\include{timeplan}
\clearpage


% \begin{appendix}
% \include{appendix_01}
% \emptydoublepage
% \include{appendix_02}
% \emptydoublepage
% \end{appendix}

%--------------------------------------------------------------
\backmatter

%Bibliographie
%bibliography
\clearpage
%\phantomsection

\bibliographystyle{plain}
\bibliography{thesis}
\cleardoublepage

%Index
%index
\markboth{Index}{Index}
\thispagestyle{plain}
%\phantomsection

%\printindex

%Satzdatum
%date of typesetting


%--------------------------------------------------------------
\end{document}

