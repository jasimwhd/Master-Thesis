%!TEX root = ./main.tex
%
% This file is part of the i10 thesis template developed and used by the
% Media Computing Group at RWTH Aachen University.
% The current version of this template can be obtained at
% <http://www.media.informatik.rwth-aachen.de/karrer.html>.

\chapter{Solution}
\label{ownwork} 
\section{Extend and develop current profiling efforts in Kylo, a data lake solution}
\label{ownwork.bigdatalake}

\subsection{Systematic process of annotating the ingested data's schema\cite{bernstein2011generic}}
\label{ownwork.nifi}
\subsection{Systematic approach of extracting, managing and exploiting metadata of the datasets' information with  ontologies
}
 \section{Analysis of the results}
 In order to make claims about our developed methodology is fool-proof, we need to analyze the results through work-flows In this work, we utilize Provenance as an infrastructure for debugging and consistency check on our results.
\subsection{Develop a dashboard containing provenance to visually to visually debug the quality of resulting data}
We perform in-depth analysis of the semantically annotated result through visualization in a dashboard, taking into two aspects in consideration: \\~\\
a) Tracking our current results against the newly scheduled run on the data source \\
b) Tracking if the process is identical from two different data stewards

\label{ownwork.standardization}

